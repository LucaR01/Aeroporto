% --------------- PROGETTAZIONE LOGICA ------------------------------------------

\newpage

\section{Progettazione Logica} %TODO: riguardarsi tutta questa parte.

\subsection{Stima del volume di dati}

\begin{tabular}{ | c  c  c | c  c  c | } %TODO: sistemare, alcune vuote sono per separare, altre perchè manca. (aggiungere quelle che mancano)
	%TODO: aggiungere i numeri del volume.
	\hline
	\textbf{Concetto} & \textbf{Tipo} & \textbf{Volume} & \textbf{Concetto} & \textbf{Tipo} & \textbf{Volume}\\
	\hline
	\textsf{\small Passeggero} & \textsf{\small E} & \textsf{\small $ 200.000$} & \textsf{\small Aereo} & \textsf{\small E} & \textsf{\small $ $}\\
	\hline
	\textsf{\small Terminal} & \textsf{\small E} & \textsf{\small $ $} & \textsf{\small Membro dell'equipaggio} & \textsf{\small E} & \textsf{\small $ $}\\
	\hline
	\textsf{\small Recarsi} & \textsf{\small R} & \textsf{\small $ $} & \textsf{\small Pilotare} & \textsf{\small R} & \textsf{\small $ $}\\
	\hline
	\textsf{\small Negozio} & \textsf{\small E} & \textsf{\small $ $} & \textsf{\small Pista} & \textsf{\small E} & \textsf{\small $ $}\\
	\hline
	\textsf{\small Comprare} & \textsf{\small R} & \textsf{\small $ $} & \textsf{\small Avvio} & \textsf{\small R} & \textsf{\small $ $}\\
	\hline
	\textsf{\small Addetto di Scalo} & \textsf{\small E} & \textsf{\small $ $} & \textsf{\small Via di rullaggio} & \textsf{\small E} & \textsf{\small $ $}\\
	\hline
	\textsf{\small Interagire} & \textsf{\small R} & \textsf{\small $ $} & \textsf{\small Ground Support Equipment} & \textsf{\small E} & \textsf{\small $ $}\\
	\hline
	\textsf{\small Addetto alla sicurezza} & \textsf{\small E} & \textsf{\small $ $} & \textsf{\small Eseguita} & \textsf{\small R} & \textsf{\small $ $}\\
	\hline
	\textsf{\small Controllare} & \textsf{\small R} & \textsf{\small $ $} & \textsf{\small } & \textsf{\small } & \textsf{\small $ $}\\
	\hline
	\textsf{\small Aiutante per disabili} & \textsf{\small E} & \textsf{\small $ $} & \textsf{\small } & \textsf{\small } & \textsf{\small $ $}\\
	\hline
	\textsf{\small Aiutare} & \textsf{\small R} & \textsf{\small $ $} & \textsf{\small } & \textsf{\small } & \textsf{\small $ $}\\
	\hline
	\textsf{\small Volo} & \textsf{\small E} & \textsf{\small $ $} & \textsf{\small } & \textsf{\small } & \textsf{\small $ $}\\
	\hline
	\textsf{\small Volare} & \textsf{\small R} & \textsf{\small $ $} & \textsf{\small } & \textsf{\small } & \textsf{\small $ $}\\
	\hline
	\textsf{\small } & \textsf{\small } & \textsf{\small $ $} & \textsf{\small } & \textsf{\small } & \textsf{\small $ $} \\
	\hline
	\textsf{\small Controllore} & \textsf{\small E} & \textsf{\small $ $} & \textsf{\small } & \textsf{\small } & \textsf{\small $ $}\\
	\hline
	\textsf{\small Controllo} & \textsf{\small R} & \textsf{\small $ $} & \textsf{\small } & \textsf{\small } & \textsf{\small $ $}\\
	\hline
	\textsf{\small Torre di Controllo} & \textsf{\small E} & \textsf{\small $ $} & \textsf{\small } & \textsf{\small } & \textsf{\small $ $}\\
	\hline
	\textsf{\small Radar} & \textsf{\small E} & \textsf{\small $ $} & \textsf{\small } & \textsf{\small } & \textsf{\small $ $}\\
	\hline
	\textsf{\small Informare} & \textsf{\small R} & \textsf{\small $ $} & \textsf{\small } & \textsf{\small } & \textsf{\small $ $}\\
	\hline
	\textsf{\small Comunicazione} & \textsf{\small R} & \textsf{\small $ $} & \textsf{\small } & \textsf{\small } & \textsf{\small $ $}\\
	\hline
	\textsf{\small Rilevazione} & \textsf{\small R} & \textsf{\small $ $} & \textsf{\small } & \textsf{\small } & \textsf{\small $ $}\\
	\hline
	\textsf{\small } & \textsf{\small } & \textsf{\small $ $} & \textsf{\small } & \textsf{\small } & \textsf{\small $ $}\\
	\hline
	\textsf{\small Compagnia Aerea} & \textsf{\small E} & \textsf{\small $ $} & \textsf{\small } & \textsf{\small } & \textsf{\small $ $}\\
	\hline
	\textsf{\small Manutenzione} & \textsf{\small E} & \textsf{\small $ $} & \textsf{\small } & \textsf{\small } & \textsf{\small $ $}\\
	\hline
	\textsf{\small } & \textsf{\small } & \textsf{\small $ $} & \textsf{\small } & \textsf{\small } & \textsf{\small $ $}\\
	\hline
	\textsf{\small Fornitori} & \textsf{\small E} & \textsf{\small $ $} & \textsf{\small } & \textsf{\small } & \textsf{\small $ $}\\
	\hline
	\textsf{\small Rifornire} & \textsf{\small R} & \textsf{\small $ $} & \textsf{\small } & \textsf{\small } & \textsf{\small $ $}\\
	\hline
	\textsf{\small Logistica} & \textsf{\small E} & \textsf{\small $ $} & \textsf{\small } & \textsf{\small } & \textsf{\small $ $}\\
	\hline
	\textsf{\small Cargo} & \textsf{\small E} & \textsf{\small $ $} & \textsf{\small } & \textsf{\small } & \textsf{\small $ $}\\
	\hline
	\textsf{\small Fornire} & \textsf{\small R} & \textsf{\small $ $} & \textsf{\small } & \textsf{\small } & \textsf{\small $ $}\\
	\hline
	\textsf{\small Mantenere} & \textsf{\small R} & \textsf{\small $ $} & \textsf{\small } & \textsf{\small } & \textsf{\small $ $}\\
	\hline
	\textsf{\small Caricare} & \textsf{\small R} & \textsf{\small $ $} & \textsf{\small } & \textsf{\small } & \textsf{\small $ $}\\
	\hline
	\textsf{\small Possedere} & \textsf{\small R} & \textsf{\small $ $} & \textsf{\small } & \textsf{\small } & \textsf{\small $ $}\\
	\hline
\end{tabular}

% ---- DESCRIZIONE DELLE OPERAZIONI PRINCIPALI E STIMA DELLA LORO FREQUENZA -----

\newpage

\subsection{Descrizione delle operazioni principale e stima della loro frequenza}

%TODO: modificare alcune operazioni per, per esempio, trovare non so, quanti dipendenti hanno età > 20; oppure quanti litri di carburante sono rimasti all'aereo.

\textsf{\small Si fornisce, di seguito, una tabella riportante la descrizione e la relativa frequenza delle operazioni principali nell'Aeroporto.}\\

\begin{tabular}{ | c c c |} %TODO: aggiungere operazioni mancanti
	\hline
	\textbf{Codice} & \textbf{Operazione} & \textbf{Frequenza} \\
	\hline
	\textsf{\small 01} & \textsf{\small Registrare un nuovo passeggero} & \textsf{\small $5000/\text{giorno}$} \\ % 200.000 / 30 = 6666.6 (-1000)
	\hline
	\textsf{\small 02} & \textsf{\small Controllare un passeggero} & \textsf{\small $ 1000 / \text{giorno} $} \\
	\hline
	\textsf{\small 03} & \textsf{\small Voli in partenza} & \textsf{\small $ 1000 / \text{mese} $} \\
	% Voli in partenza:
	% Ho preso la media del numero dei passeggeri che un aereo può portare
	% da questo sito (che mostra gli aerei più comuni) https://www.ponderweasel.com/how-many-people-can-fit-on-a-plane/?__cf_chl_managed_tk__=pmd_2mLayow7P2xTK3G5HcerVGqTNeA1cnsOiQfUmRb1Vok-1635365141-0-gqNtZGzNAxCjcnBszQh9
	% escludendo gli aerei privati e da singoli ed ho ottenuto:
	% una media pari a 
	% 525 + 220 + 189 + 290 + 451 + 366 + 172 / 7 = 316,14 -> 316
	% Se io ho scritto 5000 passeggeri / giorno, allora
	% 5000 / 316 = 15.82 -> 16 aerei per trasportare queste 5000 persone ogni giorno
	% in un mese 16 * 31 = 496 però tenendo conto che questa è una media così
	% e non tutti gli aerei possono portare tanto quanto il numero medio allora
	% metto 1000.
	\hline
	\textsf{\small 04} & \textsf{\small Voli in arrivo} & \textsf{\small $ 1000 / \text{mese} $} \\
	\hline
	\textsf{\small 05} & \textsf{\small Manutenzione di un aereo} & \textsf{\small $ 2000 / \text{mese} $} \\
	\hline
	\textsf{\small 06} & \textsf{\small Comunicazioni tra controllori e membri dell'equipaggio di un aereo} & \textsf{\small $ 1000 / \text{giorno} $} \\
	% [3 (departure) + 3 (arrival) + 3 + 3 + (eventuali emergenze) ] * (2000 / 30) = 800 / comunicazioni al giorno. oppure 15 * (2000 / 30) = 1000
	
	% oppure
	
	% 238 persone circa per aeroplano, 238 * 840 = 199.920 passeggeri
	% 840 voli / 30 giorni = 28 / voli al giorno
	
	% oppure
	
	% 130 passeggeri per volo.
	% 130 * 1540 = 200.200 passeggeri
	% 1540 voli / 30 giorni = 51 / voli giornalieri
	
	\hline
	\textsf{\small 07} & \textsf{\small Rifornimento di un aereo} & \textsf{\small $ 2000/ \text{mese} $} \\
	\hline
	\textsf{\small 08} & \textsf{\small Assunzione di nuovi addetti} & \textsf{\small $ 1000 / \text{anno}$} \\
	\hline
	\textsf{\small 09} & \textsf{\small Operazioni di check-in} & \textsf{\small $ 5000 / \text{giorno}$} \\
	\hline
	\textsf{\small 10} & \textsf{\small Aerei in pista} & \textsf{\small $ 2000 / \text{mese}$} \\
	\hline
	\textsf{\small 11} & \textsf{\small Persone che comprano prodotti ai negozi} & \textsf{\small $ 1000 / \text{giorno} $} \\ %TODO: Scrivere Acquirenti ai negozi? però non ho un'entità Acquirente.
	% Persone != Passeggeri e quindi le persone potrebbero essere di più dei passeggeri, però non tuti vanno per i negozi e ancora di meno sono quelli che comprano. Quindi quelli che visitano i negozi diciamo che sono 2000/giorno, ma quelli che comprano sono diciamo circa 1000/giorno o anche meno, magari 500.
	\hline
	\textsf{\small 12} & \textsf{\small Persone che si recano al Terminal} & \textsf{\small $ 5000 / \text{giorno} $} \\
	\hline
	\textsf{\small 13} & \textsf{\small Nuovi membri dell'equipaggio assunti da una compagnia} & \textsf{\small $ 500 / \text{anno}$} \\ %TODO: se gli addetti sono 1000 all'anno, questi saranno un po' di meno, immagino, diventare piloti richiede tempo.
	\hline
	\textsf{\small 14} & \textsf{\small Inserimento di aerei stazionati negli hangar} & \textsf{\small $ 500 / \text{mese} $} \\ % (Quite big) Boeing 747 : 5600 square feet, quite big hangar : 317000 square feet.
	% 317000 / 5600 = 56,6 -> 57
	% Gli hangar sono più di uno.
	\hline
	\textsf{\small 15} & \textsf{\small Calcolare l'età media dei passeggeri} & \textsf{\small $ 1 / \text{mese} $} \\
	\hline
	\textsf{\small 16} & \textsf{\small Ottenere il numero di aerei di una compagnia aerea.} & \textsf{\small $ 1 / \text{anno} $} \\
	\hline
	\textsf{\small 17} & \textsf{\small Numero di controllori in una Torre di Controllo} & \textsf{\small $ 1 / \text{anno} $} \\
	\hline
	\textsf{\small 18} & \textsf{\small Numero di macchinari presenti nell'Aeroporto} & \textsf{\small $ 1 / \text{mese} $} \\
	\hline
	\textsf{\small 19} & \textsf{\small Approvvigionamento dell'aereo} & \textsf{\small $ 2000 / \text{mese} $} \\
	% L'aeroporto compra il carburante e poi lo vende alle compagnie aeree.
	% altre risorse possono essere di cibo, di acqua, di servizi di catering,ecc..
	\hline
	\textsf{\small 20} & \textsf{\small Numero aerei commerciali di una compagnia aerea} & \textsf{\small $ 1/ \text{mese} $} \\
	\hline
	\textsf{\small 21} & \textsf{\small Quantità di merci trasportate in media da un aereo commerciale} & \textsf{\small $ 1/ \text{mese} $} \\
	\hline
	%TODO: add Numero di aerei di linea di una compagnia aerea ?
\end{tabular}

% ----- SCHEMI DI NAVIGAZIONE E TABELLE DEGLI ACCESSI --------------------------

%TODO: ragionarla in termini di SQL. Per questo serviranno accedere ad altre relazioni,ecc...

\newpage

\subsection{Schemi di navigazione e tabelle degli accessi}

\textsf{\small Dopo aver stimato il volume dei dati ed elencato le principali operazioni, vengono riportati qui i loro relativi schemi di navigazione. \emph{Si considerino di doppio peso gli accessi in scrittura rispetto a quelli in lettura}.}\break


%TODO: sistemare le varie opzioni e ricontrollare se basta una sola scrittura/lettura o servono di più.

% OP 1 | Registrare un nuovo passeggero

\textbf{\small OP 1 | Registrare un nuovo passeggero}\\

\begin{tabular}{ c c c c}
	%\textbf{\small OP 1 | Registrare un nuovo passeggero} & & & \\
	\hline
	\textbf{Concetto} & \textbf{Costrutto} & \textbf{Accesi} & \textbf{Tipo}\\
	\hline
	\textsf{\small Passeggero} & \textsf{\small E} & \textsf{\small 1} &  \textsf{\small S}\\
	\hline
	\textsf{\small } & \textsf{\small } & \textbf{Totale: 1S $\rightarrow 10000$/giorno} \textsf{\small } & \textsf{\small }\\ % 2 * 5000/giorno = 10000/giorno
	\hline
\end{tabular}

\vspace{.6cm}

% OP 2 | Controllare un passeggero

\textbf{\small OP 2 | Controllare un passeggero}\\
\textsf{\small Tra le migliaia di persone giornaliere, non vengono necessariamente controllate tutte, ma solo quelle di cui si ha dei sospetti o si vogliono fare ulteriori accertamenti. (con controllo non intendo il check-in).}\break

\begin{tabular}{ c c c c} %TODO: aggiungere le altre entitò e relazioni per il controllo di un passeggero
	%\textbf{\small OP 2 | Controllare un passeggero} & & & \\
	\hline
	\textbf{Concetto} & \textbf{Costrutto} & \textbf{Accesi} & \textbf{Tipo}\\
	\hline
	\textsf{\small R} & \textsf{\small Controllare} & \textsf{\small 1} &  \textsf{\small L}\\
	\hline
	\textsf{\small } & \textsf{\small } & \textbf{Totale: 1L $\rightarrow 1000$/giorno} \textsf{\small } & \textsf{\small }\\ % non si controllano mica tutte le persone, solo alcune di cui si ha dei sospetti o si vogliono fare ulteriori accertamenti.
	\hline
\end{tabular}

\vspace{.6cm}

% OP 3 | Voli in partenza

\textbf{\small OP 3 | Voli in partenza}\\

\begin{tabular}{ c c c c} %TODO: aggiungere TRATTA
	%\textbf{\small OP 3 | Voli in partenza} & & & \\
	\hline
	\textbf{Concetto} & \textbf{Costrutto} & \textbf{Accesi} & \textbf{Tipo}\\
	\hline
	\textsf{\small E} & \textsf{\small Partenza (Tratta)} & \textsf{\small 1} &  \textsf{\small S}\\
	\hline
	\textsf{\small } & \textsf{\small } & \textbf{Totale: 1S $\rightarrow 2000$/mese} \textsf{\small } & \textsf{\small }\\
	\hline
\end{tabular}

\vspace{.6cm}

% OP 4 | Voli in arrivo

\textbf{\small OP 4 | Voli in arrivo}\\

\begin{tabular}{ c c c c} %TODO: aggiungere TRATTA e VOLO
	%\textbf{\small OP 4 | Voli in arrivo} & & & \\
	\hline
	\textbf{Concetto} & \textbf{Costrutto} & \textbf{Accesi} & \textbf{Tipo}\\
	\hline
	\textsf{\small E} & \textsf{\small Destinazione (Tratta)} & \textsf{\small 1} &  \textsf{\small S}\\
	\hline
	\textsf{\small } & \textsf{\small } & \textbf{Totale: 1S $\rightarrow 2000$/mese } \textsf{\small } & \textsf{\small }\\
	\hline
\end{tabular}

\vspace{.6cm}

% OP 5 | Manutenzione di un aereo

\textbf{\small OP 5 | Manutenzione di un aereo}\\

\begin{tabular}{ c c c c} %TODO: aggiungere aereo?
	%\textbf{\small OP 5 | Manutenzione di un aereo} & & & \\
	\hline
	\textbf{Concetto} & \textbf{Costrutto} & \textbf{Accesi} & \textbf{Tipo}\\
	\hline
	\textsf{\small E} & \textsf{\small Manutenzione} & \textsf{\small 1} &  \textsf{\small S}\\
	\hline
	\textsf{\small } & \textsf{\small } & \textbf{Totale: 1S $\rightarrow 4000$/mese} \textsf{\small } & \textsf{\small }\\
	\hline
\end{tabular}

\vspace{.6cm}

% OP 6 | Comunicazioni tra controllori e membri dell'equipaggio di un aereo

\textbf{\small OP 6 | Comunicazioni tra controllori e membri dell'equipaggio di un aereo}\\

\begin{tabular}{ c c c c} %TODO: aggiungere Torre di Controllo, Controllore ed membro dell'equipaggio (non membro, ma VOLO e magari AEREO)
	%\textbf{\small OP 6 | Comunicazioni tra controllori e membri dell'equipaggio di un aereo} & & & \\
	\hline
	\textbf{Concetto} & \textbf{Costrutto} & \textbf{Accesi} & \textbf{Tipo}\\
	\hline
	\textsf{\small R} & \textsf{\small Comunicazione (Torre di Controllo)} & \textsf{\small 1} &  \textsf{\small S}\\
	\hline
	\textsf{\small } & \textsf{\small } & \textbf{Totale: 1S$\rightarrow 2000$/giorno } \textsf{\small } & \textsf{\small }\\
	\hline
\end{tabular}

\vspace{.6cm}

% OP 7 | Rifornimento di un aereo

%TODO: dalla 7 alla 14, scrivere qualche cosa.

\newpage %\pagebreak

\textbf{\small OP 7 | Rifornimento di un aereo}\\

\begin{tabular}{ c c c c}
	%\textbf{\small OP 7 | } & & & \\
	\hline
	\textbf{Concetto} & \textbf{Costrutto} & \textbf{Accesi} & \textbf{Tipo}\\
	\hline
	\textsf{\small E} & \textsf{\small GSE} & \textsf{\small 1} &  \textsf{\small S}\\
	\hline
	\textsf{\small } & \textsf{\small } & \textbf{Totale: 1S$\rightarrow 4000$/mese } \textsf{\small } & \textsf{\small }\\
	\hline
\end{tabular}

\vspace{.6cm}

% OP 8 | Assunzione di nuovi addetti

\textbf{\small OP 8 | Assunzione di nuovi addetti}\\

\begin{tabular}{ c c c c}
	%\textbf{\small OP 8 | } & & & \\
	\hline
	\textbf{Concetto} & \textbf{Costrutto} & \textbf{Accesi} & \textbf{Tipo}\\
	\hline
	\textsf{\small E} & \textsf{\small Addetto di Scalo} & \textsf{\small 1} &  \textsf{\small L}\\ %TODO: prima avevo messo lettura, ma un nuovo addetto mi sembra sia più una scrittura che una lettura
	\hline
	\textsf{\small } & \textsf{\small } & \textbf{Totale: 1S$\rightarrow 2000$/anno } \textsf{\small } & \textsf{\small }\\ % 2 * 1000 = 2000
	\hline
\end{tabular}

\vspace{.6cm}

% OP 9 | 

\textbf{\small OP 9 | Operazioni di check-in }\\

\begin{tabular}{ c c c c} %TODO: Aggiungere Addetto di Scalo.
	%\textbf{\small OP 9 | } & & & \\
	\hline
	\textbf{Concetto} & \textbf{Costrutto} & \textbf{Accesi} & \textbf{Tipo}\\
	\hline
	\textsf{\small R} & \textsf{\small Interagire} & \textsf{\small 1} &  \textsf{\small L}\\
	\hline
	\textsf{\small } & \textsf{\small } & \textbf{Totale: 1L$\rightarrow 5000$/giorno } \textsf{\small } & \textsf{\small }\\
	\hline
\end{tabular}

\vspace{.6cm}

% OP 10 | 

\textbf{\small OP 10 | Aerei in pista}\\

\begin{tabular}{ c c c c} %TODO: Aggiungere Aereo e relazione con pista
	%\textbf{\small OP 10 | } & & & \\
	\hline
	\textbf{Concetto} & \textbf{Costrutto} & \textbf{Accesi} & \textbf{Tipo}\\
	\hline
	\textsf{\small E} & \textsf{\small Pista} & \textsf{\small 1} &  \textsf{\small L}\\
	\hline
	\textsf{\small } & \textsf{\small } & \textbf{Totale: 1L$\rightarrow 2000$/mese } \textsf{\small } & \textsf{\small }\\
	\hline
\end{tabular}

\vspace{.6cm}

% OP 11 | Acquirenti ai negozi

\textbf{\small OP 11 | Acquirenti ai negozi}\\ %TODO: rimettere Persone che comprano prodotti ai negozi (o nei negozi).

\begin{tabular}{ c c c c}
	%\textbf{\small OP 11 | } & & & \\
	\hline
	\textbf{Concetto} & \textbf{Costrutto} & \textbf{Accesi} & \textbf{Tipo}\\
	\hline
	\textsf{\small E} & \textsf{\small Negozio} & \textsf{\small 1} &  \textsf{\small L}\\
	\hline
	\textsf{\small } & \textsf{\small } & \textbf{Totale: 1L$\rightarrow 1000$/giorno } \textsf{\small } & \textsf{\small }\\
	\hline
\end{tabular}

\vspace{.6cm}

% OP 12 | Passeggeri che si recano al Terminal
%TODO: meglio Persone o Passeggeri, teoricamente è più corretto Passeggeri.

\textbf{\small OP 12 | Persone che si recano al Terminal}\\

\begin{tabular}{ c c c c} %TODO: aggiungere Terminal
	%\textbf{\small OP 12 | } & & & \\
	\hline
	\textbf{Concetto} & \textbf{Costrutto} & \textbf{Accesi} & \textbf{Tipo}\\
	\hline
	\textsf{\small R} & \textsf{\small Recarsi} & \textsf{\small 1} &  \textsf{\small S}\\
	\hline
	\textsf{\small } & \textsf{\small } & \textbf{Totale: 1S$\rightarrow 10000$/giorno } \textsf{\small } & \textsf{\small }\\ % 2 * 5000 = 10000
	\hline
\end{tabular}

\vspace{.6cm}

% OP 13 | Nuovi membri dell'equipaggio assunti da una compagnia

\textbf{\small OP 13 | Nuovi membri dell'equipaggio assunti da una compagnia}\\

\begin{tabular}{ c c c c} %TODO: Aggiungere relazione Assumere e Membro dell'equipaggio.
	%\textbf{\small OP 13 | } & & & \\
	\hline
	\textbf{Concetto} & \textbf{Costrutto} & \textbf{Accesi} & \textbf{Tipo}\\
	\hline
	\textsf{\small E} & \textsf{\small Compagnia aerea} & \textsf{\small 1} &  \textsf{\small S}\\
	\hline
	\textsf{\small } & \textsf{\small } & \textbf{Totale: 1S$\rightarrow 1000$/anno } \textsf{\small } & \textsf{\small }\\ % 2 * 500 = 1000
	\hline
\end{tabular}

\vspace{.6cm}

% OP 14  | Inserimento di aerei stazionati negli Hangar

\textbf{\small OP 14 | Inserimento di aerei stazionati negli Hangar}\\

\begin{tabular}{ c c c c}
	%\textbf{\small OP 14 | } & & & \\
	\hline
	\textbf{Concetto} & \textbf{Costrutto} & \textbf{Accesi} & \textbf{Tipo}\\
	\hline
	\textsf{\small E} & \textsf{\small Hangar} & \textsf{\small 1} &  \textsf{\small S}\\
	\hline
	\textsf{\small } & \textsf{\small } & \textbf{Totale: 1S$\rightarrow 1000$/mese } \textsf{\small } & \textsf{\small }\\ % 2 * 500 == 1000
	\hline
\end{tabular}

\vspace{.6cm}

% OP 15 | Calcolare l'età media dei passeggeri

\textbf{\small OP 15 | Calcolare l'età media dei passeggeri}\\

\textsf{\small Ai fini statistici, per capire quali persone viaggiano di più, a che età e dove, ecc.. Viene registrata e calcolata l'età media delle persone e vari altri dati.}\break

\begin{tabular}{ c c c c}
	%\textbf{\small OP 15 | } & & & \\
	\hline
	\textbf{Concetto} & \textbf{Costrutto} & \textbf{Accesi} & \textbf{Tipo}\\
	\hline
	\textsf{\small E} & \textsf{\small Passeggero} & \textsf{\small 1} &  \textsf{\small L}\\
	\hline
	\textsf{\small E} & \textsf{\small Passeggero} & \textsf{\small 1} &  \textsf{\small S}\\ %TODO: per la scrittura della media.
	\hline
	\textsf{\small } & \textsf{\small } & \textbf{Totale: 1L + 1S$\rightarrow 2$/mese } \textsf{\small } & \textsf{\small }\\
	\hline
\end{tabular}

\vspace{.6cm}

% OP 16  | Ottenere il numero di aerei di una compagnia aerea

\textbf{\small OP 16 | Ottenere il numero di aerei di una compagnia aerea}\\

\begin{tabular}{ c c c c}
	%\textbf{\small OP 16 | } & & & \\
	\hline
	\textbf{Concetto} & \textbf{Costrutto} & \textbf{Accesi} & \textbf{Tipo}\\
	\hline
	\textsf{\small E} & \textsf{\small Compagnia Aerea} & \textsf{\small 1} &  \textsf{\small L}\\
	\hline
	\textsf{\small } & \textsf{\small } & \textbf{Totale: $\rightarrow $/ } \textsf{\small } & \textsf{\small }\\
	\hline
\end{tabular}

\vspace{.6cm}

% OP 17  | Numero di controllori in una Torre di Controllo

\textbf{\small OP 17 | Numero di controllori in una Torre di Controllo}\\

\begin{tabular}{ c c c c}
	%\textbf{\small OP 17 | } & & & \\
	\hline
	\textbf{Concetto} & \textbf{Costrutto} & \textbf{Accesi} & \textbf{Tipo}\\
	\hline
	\textsf{\small E} & \textsf{\small Torre di Controllo} & \textsf{\small 1} &  \textsf{\small L}\\
	\hline
	\textsf{\small } & \textsf{\small } & \textbf{Totale: 1L$\rightarrow 1$/anno } \textsf{\small } & \textsf{\small }\\
	\hline
\end{tabular}

\vspace{.6cm}

% OP 18  | Numero di macchinari presenti nell'Aeroporto

\textbf{\small OP 18 | Numero di macchinari presenti nell'Aeroporto}\\

\textsf{\small Per il controllo,la manutenzione dei vari macchinari e per la corretta esecuzione delle normali attività aeroportuali.}\break

\begin{tabular}{ c c c c}
	%\textbf{\small OP 18 | } & & & \\
	\hline
	\textbf{Concetto} & \textbf{Costrutto} & \textbf{Accesi} & \textbf{Tipo}\\
	\hline
	\textsf{\small E} & \textsf{\small GSE} & \textsf{\small 1} &  \textsf{\small L}\\
	\hline
	\textsf{\small } & \textsf{\small } & \textbf{Totale: 1L$\rightarrow 1$/mese } \textsf{\small } & \textsf{\small }\\
	\hline
\end{tabular}

\vspace{.6cm}

% OP 19  | Approvvigionamento dell'aereo

\textbf{\small OP 19 | Approvvigionamento dell'aereo}\\

\textsf{\small Rifornire l'aereo di varie risorse di primaria importanza per il suo funzionamento, non soltanto del carburante, ma anche di cibo, acqua, servizi di catering,ecc.. per il normale svolgimento delle operazioni e per il soddisfacimento dei bisogni dei passeggeri.}\break

\begin{tabular}{ c c c c}
	%\textbf{\small OP 19 | } & & & \\
	\hline
	\textbf{Concetto} & \textbf{Costrutto} & \textbf{Accesi} & \textbf{Tipo}\\
	\hline
	\textsf{\small E} & \textsf{\small GSE} & \textsf{\small 1} &  \textsf{\small S}\\
	\hline
	\textsf{\small } & \textsf{\small } & \textbf{Totale: 1S$\rightarrow 4000$/ mese} \textsf{\small } & \textsf{\small }\\
	\hline
\end{tabular}

\vspace{.6cm}

% OP 20  | Numero aerei commerciali di una compagnia aerea

\textbf{\small OP 20 | Numero aerei commerciali di una compagnia aerea}\\

\begin{tabular}{ c c c c}
	%\textbf{\small OP 20 | } & & & \\
	\hline
	\textbf{Concetto} & \textbf{Costrutto} & \textbf{Accesi} & \textbf{Tipo}\\
	\hline
	\textsf{\small E} & \textsf{\small Compagnia Aerea} & \textsf{\small 1} &  \textsf{\small L}\\
	\hline
	\textsf{\small } & \textsf{\small } & \textbf{Totale: 1L$\rightarrow 1$/mese } \textsf{\small } & \textsf{\small }\\
	\hline
\end{tabular}

\vspace{.6cm}

% OP 21  | Quantità di merci trasportate in media da un aereo commerciale

\textbf{\small OP 21 | Quantità di merci trasportate in media da un aereo commerciale}\\

\textsf{\small Per tenere traccia di quante merci sono state vendute e trasportate da una compagnia aerea.}\break

\begin{tabular}{ c c c c}
	%\textbf{\small OP 21 | } & & & \\
	\hline
	\textbf{Concetto} & \textbf{Costrutto} & \textbf{Accesi} & \textbf{Tipo}\\
	\hline
	\textsf{\small E} & \textsf{\small Compagnia Aerea} & \textsf{\small 1} &  \textsf{\small L}\\
	\hline
	\textsf{\small } & \textsf{\small } & \textbf{Totale: 1L$\rightarrow 1$/mese } \textsf{\small } & \textsf{\small }\\
	\hline
\end{tabular}

% --------------- RAFFINAMENTO DELLO SCHEMA ------------------------------------

\newpage

\subsection{Raffinamento dello schema}

%\subsubsection{Eliminazione gerarchie}
\textbf{Eliminazione delle gerarchie}\\
%TODO: quel superflue è corretto metterlo? o sarebbe da mettere nelle Ridondanze?
\textsf{\small Per quanto riguarda l'eliminazione delle gerarchie, ho deciso di adottare il collasso verso l'alto per la maggior parte di esse, come soluzione del problema, mentre per altre il collasso verso il basso.}\break 

% Collasso verso l'alto: Accorpare le entità figlie nel genitore

% Collasso verso il basso: Accorpare il genitore nelle entità figlie

\textsf{\small Ho individuato 8 gerarchie che ho collassato verso l'alto e aggiunto loro gli attributi \emph{Tipologia} o \emph{Ruolo}, ovvero \emph{Componente dell'aereo}, \emph{Membro dell'equipaggio}, \emph{Radar}, \emph{Controllore}, \emph{Addetto di scalo}, \emph{Negozio}, \emph{Ground Support Equipment},\emph{Tratta}.}\\

%TODO: Verso l'alto anche per Persona.
%TODO: Verso il basso solo per Terminal
\textsf{\small Per quanto riguarda \emph{Persona} invece ho ritenuto più opportuno adottare il collasso verso il basso, in quanto gli accessi alle entità figlie sono distinti.}\break
%TODO: collasso verso l'alto anche per Persona?

%TODO: metterlo nei ridondanti?

\textbf{Eliminazione degli attributi compositi}

\textsf{\small Non è stato fatto uso di attributi composti perciò non è stato dovuto assestare in alcun modo la faccenda.}\break %assestato, situazione/contesto.

\textbf{Scelte delle chiavi primarie}

\textsf{\small Nello schema proposto sono già presenti le chiavi primarie per tutte le entità, identificate ciascuna da un proprio codice univoco.}\break

\textbf{Eliminazione degli identificatori esterni}

\textsf{\small A seguito delle analisi compiute è stato determinato di rimuovere le seguenti relazioni:}\\ %deciso

% SOLO RELAZIONI
\begin{itemize} %TODO: forse alcune sono superflue, perchè le ho già cambiate nello schema.
	%TODO: rimuovere hangar e aereo, non nello schema, ma qui, aereo e via di rullaggio. 
	%\item \textsf{\small Relazione \emph{appartenere} tra \textbf{PASSEGGERO} e \textbf{BAGAGLIO}, importando \emph{Num. Bagagli} in \textbf{PASSEGGERO}.} %TODO: attributo numero valigie al posto delle relazione PASSEGGERO e BAGAGLIO (o VALIGIA)
	\item \textsf{\small Relazione \emph{appartenere} tra \textbf{Passeggero} e \textbf{Bagaglio}, importando \emph{CodBagaglio} in \textbf{Persona}.}
	%\item \textsf{\small Associazione \emph{disporre} tra \textbf{AEREO} e \textbf{COMPAGNIA AEREA}, introducendo \emph{Num. Aerei} in \textbf{COMPAGNIA AEREA}.} %TODO: forse questa l'ho già cancellata nello schema?!
	\item \textsf{\small \emph{Operare} tra \textbf{Controllore} e \textbf{Torre di Controllo}, usando invece un attributo \emph{CodControllore} in \textbf{Torre di Controllo}.}
	%\item \textsf{\small } %TODO: Tra AEREO e COMPONENTE DELL'AEREO potrei mettere un attributo Num. Componenti
	%\item \textsf{\small \emph{Operare} tra \textbf{CONTROLLORE} e \textbf{TORRE DI CONTROLLO}, usando invece un attributo \emph{Num. Controllori} in \textbf{TORRE DI CONTROLLO}.} %TODO: questo attributo Num. Controllori ce l'ho già.
	%\item \textsf{\small \emph{Risiedere} tra \textbf{AEREO} ed \textbf{HANGAR} , usufruendo di un attributo \emph{Hangar[0:1]} in \textbf{AEREO} per sapere se l'aereo si trova nell'hangar e anche in quale.} %TODO: Al posto della relazione RISIEDERE tra AEREO ed HANGAR, magari potrei mettere direttamente un attributo in AEREO per sapere dove risiede e se sta risiedendo lì. tipo [0:1]
	
	%\item \textsf{\small \emph{Far parte} tra \textbf{AEREO} e \textbf{MEMBRO DELL'EQUIPAGGIO}, potrei avvalermi di un attributo \emph{Num. Equipaggio} nell'entià \textbf{AEREO}.} %TODO: AEREO e MEMBRO DELL'EQUIPAGGIO, piuttosto potrei avere un attributo Num.Equipaggio (che in realtà già ho). 
	%\item \textsf{\small \emph{Assumere} tra \textbf{COMPAGNIA AEREA} e \textbf{MEMBRO DELL'EQUIPAGGIO}, utilizzando un attributo \emph{Num. Dipendenti} o \emph{Num. Personale}.}%TODO: In realtà, ho già questo attributo in COMPAGNIA AEREA.
	
	%\item \textsf{\small }%TODO: relazioni OPERATA ed ESEGUITA in una GERARCHIA di MANUTENZIONE, (mah insomma)
	\item \textsf{\small \emph{Servire} tra \textbf{Persona} e \textbf{Servizio Clienti} importando \emph{Codice Fiscale} in \textbf{Servizio Clienti}.}
	\item \textsf{\small \emph{Visitare} tra \textbf{Persona} e \textbf{Negozio} importando \emph{Codice Fiscale} in \textbf{Negozio}.}
	\item \textsf{\small \emph{Recarsi} tra \textbf{Terminal} e \textbf{Persona}, importando \emph{CodiceFiscale} in \textbf{Terminal}.}
	\item \textsf{\small \emph{Intervenire} tra \textbf{Centro di Controllo d'Aerea} e \textbf{Persona}, importando \emph{CodiceFiscale} in \textbf{Centro di controllo d'Area}.}
	\item \textsf{\small \emph{Informare} tra \textbf{Persona} e \textbf{Radar}, importando \emph{CodRadar} in \textbf{Persona}.}
	\item \textsf{\small \emph{Contattare} tra \textbf{Persona} e \textbf{Soccorsi}, trasferendo \emph{CodSoccorso} in \textbf{Persona}.}
	\item \textsf{\small \emph{Appartenere} tra \textbf{Persona} e \textbf{Bagaglio}, trasportando \emph{CodBagaglio} in \textbf{Persona}.}
	\item \textsf{\small \emph{Accertare} tra \textbf{Persona} e \textbf{Bagaglio}, importando \emph{CodBagaglio} in \textbf{Persona}.} %TODO: Questo serve?
	\item \textsf{\small \emph{Adempiere} tra \textbf{Persona} e \textbf{Mantenimento}, importando \emph{CodiceFiscale} in \textbf{Mantenimento}. }
	\item \textsf{\small \emph{Rifornire} tra \textbf{Persona} e \textbf{Logistica}, trasferendo \emph{CodiceFiscale} in \textbf{Logistica}. }
	\item \textsf{\small \emph{Prendere} tra \textbf{Volo} e \textbf{Persona}, inserendo \emph{CodiceFiscale} in \textbf{Volo}.}
	\item \textsf{\small \emph{Comunicazione} tra \textbf{Torre di Controllo} e \textbf{Persona}, aggiungendo \emph{CodiceFiscale} in \textbf{Torre di Controllo}.}
	\item \textsf{\small \emph{Possedere} tra \textbf{Compagnia Aerea} e \textbf{Negozio}, importando \emph{CodNegozio} in \textbf{Compagnia Aerea}.}
	\item \textsf{\small \emph{Costituire} tra \textbf{Gate} e \textbf{Terminal}, trasferendo \emph{CodTerminal} in \textbf{Gate}.}
	\item \textsf{\small \emph{Rilevazione} tra \textbf{Radar} e \textbf{Aereo}, inserendo \emph{CodAereo} in \textbf{Radar}.}
	\item \textsf{\small \emph{Mantenimento} tra \textbf{Ground Support Equipment} e \textbf{Mantenimento}, immettendo \emph{CodMacchinario} in \textbf{Mantenimento}.}
	\item \textsf{\small \emph{Detenere} tra \textbf{Aereo} e \textbf{Volo}, incorporando \emph{CodAereo} in \textbf{Volo}.}
	\item \textsf{\small \emph{Servizio} tra \textbf{Tratta} e \textbf{Volo}, allegando \emph{CodTratta} in \textbf{Volo}.}
	\item \textsf{\small \emph{Preservare} tra \textbf{Aereo} e \textbf{Mantenimento}, conglobando \emph{CodAereo} in \textbf{Mantenimento}.}
	\item \textsf{\small \emph{Transitare} tra \textbf{Aereo} e \textbf{Via di Rullaggio}, accorpando \emph{CodAereo} in \textbf{Via di Rullaggio}.}
	\item \textsf{\small \emph{Avviarsi} tra \textbf{Aereo} e \textbf{Pista},inglobando \emph{CodPista} in \textbf{Aereo}.}
	\item \textsf{\small \emph{Comporre} tra \textbf{Aereo} e \textbf{Componente dell'aereo}, inserendo \emph{CodAereo} in \textbf{Componente dell'aereo}.} % prima era l'incontrario. ma non si può per via delle cardinalità
	\item \textsf{\small \emph{Caricare} tra \textbf{Cargo} ed \textbf{Aereo},importando \emph{CodAereo} in \textbf{Cargo}.}
	\item \textsf{\small \emph{Mantenere} tra \textbf{Persona} e \textbf{Aereo}, aggiungendo \emph{CodAereo} in \textbf{Persona}.}
	\item \textsf{\small \emph{Risiedere} tra \textbf{Aereo} ed \textbf{Hangar}, trasferendo \emph{CodHangar} in \textbf{Aereo}.}
	\item \textsf{\small \emph{Tutelare} tra \textbf{Compagnia Aerea} e \textbf{Persona}, importando \emph{CodiceFiscale} in \textbf{Compagnia Aerea}.}
	\item \textsf{\small \emph{Fornire} tra \textbf{Logistica} e \textbf{Cargo}, conglobando \emph{CodLogistica} in \textbf{Cargo}.}
	\item \textsf{\small \emph{Ottenere} tra \textbf{Compagnia Aerea} e \textbf{Logistica}, incorporando \emph{CodCompagnia} in \textbf{Logistica}.}
	\item \textsf{\small \emph{Assicura} tra \textbf{Compagnia Aerea} e \textbf{Assicurazione}, introducendo \emph{CodAssicurazione} in \textbf{Compagnia Aerea}.}
	%\item \textsf{\small \emph{} tra \textbf{} e \textbf{}, \emph{} in \textbf{}.}
	%\item \textsf{\small \emph{} tra \textbf{} e \textbf{}, \emph{} in \textbf{}.}
	%\item \textsf{\small \emph{} tra \textbf{} e \textbf{}, \emph{} in \textbf{}.}
\end{itemize}

% --------------- ANALISI DELLE RIDONDANZE -------------------------------------

\newpage

\subsection{Analisi delle ridondanze}

\textsf{\small Dopo le svariate analisi, son state rilevate le seguenti ridondanze: }\break

%TODO: rimuovere quella sulla tratta (già rimossa nello schema)
%TODO: num bagagli perchè sta già prima
\begin{itemize}
	%\item \textsf{\small Le entità \textbf{TRATTA} e \textbf{ORARIO} al posto di porle come attributo composto in \textbf{VOLO}.}\\%TODO: entità TRATTA e ORARIO e al posto metterli come attributi composti in VOLO. %TODO: io questa la toglierei.
	\item \textsf{\small L'entità \textbf{BAGAGLIO} che si potrebbe sostituire da un attributo \emph{Num. Bagagli} in \textbf{PASSEGGERO}.}\\%TODO: mettere BAGAGLIO direttamente come attributo in PASSEGGERO, o meglio come attributo "Num.Bagagli" o "Num.Valigie".
	%\item \textsf{\small Avrei potuto mettere l'attributo \emph{Ruolo[0:1]} direttamente in \textbf{PERSONA}, al posto di metterla solo in alcune entità.}\\%TODO: Attributo: Ruolo in alcuni, al posto metterlo una singola volta come attributo in PERSONA.
	%TODO: rimuovere RUOLO perchè non serve effettivamente, sappiamo già che ruolo hanno.
	%TODO: Persona l'ho tolta dalle gerarchie.
	%\item \textsf{\small Anzichè la relazione \emph{RECARSI} e l'entità \textbf{GATE}, avrei potuto far uso dell'attributo \emph{Gate} direttamente in \textbf{PASSEGGERO}.}\\%TODO: forse al posto della relazione RECARSI e dell'entità GATE. Mettere direttamente come attributo in PASSEGGERO il numero di gate a cui si deve recare.
	%\item \textsf{\small }
\end{itemize}

%TODO: gerarchia PERSONALE DELL'AEROPORTO? oppure come entità al posto di tutte le entità: ADDETTO DI SCALO, AIUTANTE PER DISABILI E ADDETTO ALLA SICUREZZA, SERVIZIO CLIENTI, ecc...
%TODO: Oppure come detto, PERSONALE DELL'AEROPORTO come Gerarchia e ci mettiamo RUOLO, oppure ruolo non serve in primo luogo.

%TODO: entità TIPOLOGIA per AEREO?

%TODO: relazioni OPERATA ed ESEGUITA in una GERARCHIA di MANUTENZIONE

%TODO: attributo "persona da aiutare" in Aiutante per disabili

%TODO: ridondanze sulle operazioni

% --------------- TRADUZIONE DI ENTITA' ED ASSOCIAZIONI IN RELAZIONI -----------

\newpage

\subsection{Traduzione di entità ed associazioni in relazioni}

%TODO: da fare riguardante lo schema relazionale/logico finale
%TODO: farlo dopo aver fatto quello.

\begin{itemize}
	\item \textbf{\small } \textsf{\small ()}
	\item \textbf{\small } \textsf{\small ()}
	\item \textbf{\small } \textsf{\small ()}
\end{itemize}

% --------------- SCHEMA RELAZIONALE FINALE ------------------------------------

\newpage

\subsection{Schema relazionale finale} % logico

%TODO: aggiungere l'immagine dello schema logico finale.

% --------------- TRASFORMAZIONE DELLE OPERAZIONI IN QUERY SQL -----------------

\newpage

% oppure riproposizione delle operazioni in query sql
\subsection{Trasformazione delle operazioni in query SQL}

% OP 1 | Registrare un nuovo passeggero

\textbf{\small OP 1 | Registrare un nuovo passeggero}
%TODO: usare package per i linguaggi di programmazione, in questo caso SQL

%TODO: aggiungere le altre operazioni.

% OP 2 |Controllare un passeggero